\documentclass[12pt]{article}

\title{Problem Set 1: The Gamma Function}
\author{MATH E-158: Introduction to Bayesian Inference}
\author{David Shaub}
\date{Due September 11, 2017}


\usepackage{amsmath}
\usepackage{amsthm}
\usepackage{amssymb}

\newtheorem{theorem}{Theorem}[section]

\theoremstyle{definition}
\newtheorem{definition}{Definition}[section]

\renewcommand\qedsymbol{$\blacksquare$}
	
\begin{document}
	
	\maketitle





\section*{Problem 1}

This problem tests your understanding of the fundamental recurrence relation for the gamma function.

\subsection*{Problem Statement}

\noindent
{\bf Part (a)}\ What is the value of $\Gamma(11)$?

\bigskip
\noindent
{\bf Part (b)}\ What is the value of $\Gamma(4.5)$?

\bigskip
\noindent
{\bf Part (c)}\ What is the value of $\displaystyle \frac{ \Gamma(7) }{\Gamma(5)} \cdot \frac{\Gamma(6.7)}{\Gamma(8.7)}?$

\subsection*{Problem Solution}

\noindent
{\bf Part (a)}
\begin{align*}
\Gamma(n) &= (n - 1)!\\
\Gamma(11) &= (11 - 1)!\\
           &= \prod_{i=1}^{10} i = 3628800
\end{align*}

\newpage
\subsection*{Problem 1 (continued)}
\noindent
{\bf Part (b)}
\begin{align*}
\Gamma(4.5) &= 3.5 \cdot \Gamma(3.5)\\
            &= (3.5 \cdot 2.5) \cdot \Gamma(2.5)\\
            &= (3.5 \cdot 2.5 \cdot 1.5) \cdot \Gamma(1.5)\\
            &= (3.5 \cdot 2.5 \cdot 1.5 \cdot 0.5) \cdot \Gamma(\frac{1}{2})\\
            &= 6.5625 \cdot \sqrt{\pi}\\
            &\approxeq 11.6317
\end{align*}

\vspace{2in}
\noindent
{\bf Part (c)}
\begin{align*}
\frac{ \Gamma(7) }{\Gamma(5)} \cdot \frac{\Gamma(6.7)}{\Gamma(8.7)} &= \frac{6!}{4!} \cdot \frac{\Gamma(6.7)}{\Gamma(8.7)}\\
&= \frac{5 \cdot 6}{1} \cdot \frac{1}{6.7 \cdot 7.7}\\
&\approxeq 0.5815
\end{align*}

\newpage
\section*{Problem 2}

This is another problem involving the recurrence relation. You must show the recurrence in your submitted work. You can check your answer using a computer (e.g.\ Microsoft Excel) but don't just tell us what the number is -- you must explicitly show us how you got it.

\subsection*{Problem Statement}

Using the recurrence relation, we can actually define the gamma function for negative numbers, as long as they are not integers (i.e.\ whole numbers). Calculate the value of $\Gamma(-2.5)$.

\subsection*{Problem Solution}
\begin{align*}
\Gamma(n) &= (n - 1) \cdot \Gamma(n - 1)\\
\Gamma(n - 1) &= \frac{\Gamma(n)}{(n - 1)}\\
\Gamma(\frac{1}{2}) &= \sqrt{\pi}\\
\end{align*}
\begin{align*}
\Gamma(-2.5) &= \frac{\Gamma(-1.5)}{-2.5}\\
&= \frac{\frac{\Gamma(-0.5)}{-1.5}}{-2.5}\\
&= \frac{\frac{\frac{\Gamma(0.5)}{-0.5}}{-1.5}}{-2.5}\\
&\approxeq -0.9453
\end{align*}

\newpage
\section*{Problem 3}

This problem tests your understanding of the relation between the gamma function and the factorial function.

\subsection*{Problem Statement}

{\bf Part (a)}\ \ The binomial coefficient is defined for non-negative integers $r$ and $k$ with $r \geq k$ as
$$
{r \choose k} = \frac{r!}{k! \cdot (r - k)!}
$$
Use the gamma function to define a generalized form of the binomial coefficient, where $r$ is still non-negative but is not necessarily an integer. Your generalized form should return the same answer as before when $r$ is an integer. Note that $k$ is still restricted to integer values.

\bigskip
\noindent
{\bf Part (b)}\ \ What is the value of
$
\displaystyle { 7.5 \choose 3}
$?

\subsection*{Problem Solution}

\noindent
{\bf Part (a)}
\begin{equation}
{r \choose k} = \frac{\Gamma(r + 1)}{\Gamma(k + 1) \cdot \Gamma(r - k + 1)}
\end{equation}

\newpage
\subsection*{Problem 3 (continued)}


\vspace{0.5in}
\noindent
{\bf Part (b)}
\begin{align*}
{7.5 \choose 3} &= \frac{\Gamma(8.5)}{\Gamma(4) \cdot \Gamma(7.5 - 3 + 1)}\\
&= \frac{\Gamma(8.5)}{3 \cdot 2 \cdot \Gamma(5.5)}\\
&= \frac{\Gamma(8.5)}{6 \cdot \Gamma(5.5)}\\
&= \frac{7.5 \cdot 6.5 \cdot 5.5}{6}\\
&= 44.6875
\end{align*}

\newpage
\section*{Problem 4}

Now it's time for your first integral! You have to use integration by parts for this; it is very similar to what we did in lecture.

\subsection*{Problem Statement}

Calculate the value of $\Gamma(3)$ directly from the definition i.e.\ evaluate the integral
$$
\int_0^\infty t^2 e^{-t} \cdot dt
$$
You can use the fact that
$$
\int_0^\infty t e^{-t} \cdot dt = 1
$$

\subsection*{Problem Solution}
\begin{align*}
\Gamma(3) &= \int_0^\infty t^2 e^{-t} \cdot dt\\\\
u &= t^2 \quad\Rightarrow\quad du = 2t \cdot dt\\
dv &= e^{-t} \cdot dt \quad\Rightarrow\quad v = -e^{-t}\\\\
\Gamma(3) &=  t^{2} \cdot (-e^{-t})\biggr\rvert_0^{\infty} - \int_0^\infty -2te^{-t} \cdot dt\\
&= 0 + 2\int_0^\infty te^{-t} \cdot dt\\
&= 2
\end{align*}

\newpage
\subsection*{Problem 4 (continued)}


\newpage
\section*{Problem 5}

This integral will test your skills in substitution.

\subsection*{Problem Statement}

Evaluate the integral
$$
\int_0^\infty t^4 e^{-3t} \cdot dt
$$
Hint: this looks very similar to a gamma function, but it's not quite the same. Make a substitution so that this does become a gamma function, and then evaluate that.

\subsection*{Problem Solution}
\begin{align*}
u = 3t \quad&\Rightarrow\quad t = \frac{u}{3}\\
du = 3dt \quad&\Rightarrow\quad dt = \frac{du}{3}\\\\
\int_0^\infty t^4 e^{-3t} \cdot dt &= \int_0^\infty \biggr(\frac{u}{3}\biggr)^4 e^{-u} \frac{1}{3}\cdot du\\
&= \frac{1}{243}\int_0^\infty u^4 e^{-u}\cdot du\\
&= \frac{1}{243} \Gamma(5)\\
& = \frac{24}{243} = \frac{8}{81}
\end{align*}




\newpage
\section*{Problem 6}

This problem is very similar to the previous one; you still have to do a substitution, but now you have a little more algebra to contend with.

\subsection*{Problem Statement}

Derive an algebraic expression for the integral
$$
\int_0^\infty t^k e^{-\beta t} \cdot dt
$$

\subsection*{Problem Solution}
\begin{align*}
u = \beta t \quad&\Rightarrow\quad t = \frac{u}{\beta}\\
du = \beta dt \quad&\Rightarrow\quad dt = \frac{du}{\beta}\\\\
\int_0^\infty t^k e^{-\beta t} \cdot dt &= \int_0^\infty \biggr(\frac{u}{\beta}\biggr)^k e^{-u} \frac{1}{\beta}\cdot du\\
&= \frac{1}{\beta^{k + 1}}\int_0^\infty u^k e^{-u}\cdot du\\
&= \frac{1}{\beta^{k + 1}} \Gamma(k + 1)\\
\end{align*}



\newpage
\section*{Problem 7 (Graduate)}

This problem also involves substitution, but it's a little more advanced.

\subsection*{Problem Statement}

Evaluate the integral
$$
\int_0^\infty t^{-\alpha- 1} e^{-\beta/t} \cdot dt
$$
Hint: again, this looks kind of like a gamma function, but not quite. Make a substitution to turn it into a gamma function, and then evaluate. There is nothing particularly tricky about this problem, but you must be careful about every step. Make sure you pay attention to the differential element and the limits of integration.

\subsection*{Problem Solution}

\newpage
\subsection*{Problem 7 (continued)}



\newpage
\section*{Problem 8 (Graduate)}

This is the most advanced integration problem in the problem set, and possibly in the whole course. Once again, it's just a question of getting the right substitution, but as with the previous problem you have to be very careful to do everything right.

\subsection*{Problem Statement}

A standard result is
$$
\int_{-\infty}^\infty e^{-t^2} \cdot dt = \sqrt{\pi}
$$
Use this to derive the result
$$
\Gamma \left ( \frac{1}{2} \right ) = \sqrt{\pi}
$$
Hint: the integral goes from $-\infty$ to $+\infty$, but the gamma function is defined by an integral from $0$ to $+\infty$. So the first step is to obtain a new integral that has the appropriate range of integration. You actually can do this with a conceptual argument; no algebra manipulation required. Then you'll find that there is again something in the exponent that doesn't quite fit with the gamma function; if you make the right substitution, you'll find that you do indeed end up with the desired gamma function.

\subsection*{Problem Solution}
\begin{align*}
\Gamma \left ( \frac{1}{2} \right ) &= \int_0^\infty t^{-\frac{1}{2}}e^{-t} \cdot dt\\
&= 2 \int_0^\infty \frac{1}{2}t^{-\frac{1}{2}}e^{-t} \cdot dt\\\\
u &= \sqrt{t} \quad\Rightarrow\quad du = \frac{1}{2}t^{-\frac{1}{2}} \cdot dt\\
\end{align*}

\newpage
\subsection*{Problem 8 (continued)}
Using this, we perform substitution for $u$ and $du$ and arrive at a formulation similar to our standard result but with different integration bounds:
\begin{align*}
\Gamma \left ( \frac{1}{2} \right ) &= 2 \int_0^\infty \frac{1}{2}t^{-\frac{1}{2}}e^{-t} \cdot dt\\
&= 2\int_0^\infty e^{-u^2}\cdot du\\
\end{align*}
Since $f(u) = e^{-u^2}$ is symmetric about the y-axis, we can rewrite the standard result with the integration bounds from zero to infinity and conclude our proof:
\begin{align*}
\int_{-\infty}^\infty e^{-u^2} \cdot du = 2\int_0^\infty e^{-u^2} \cdot du= \sqrt{\pi}
\end{align*}

\end{document}
